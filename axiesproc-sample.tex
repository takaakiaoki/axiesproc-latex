% axiesproc クラスを利用します.
% このクラスは jsarticle に依存し, ひいては pLaTeX2e を要求します
\documentclass{axiesproc}

% 必要に応じ, 適宜パッケージを追加してください.
% usepackage{amsmath}

% 和文タイトル
\title{大学ICT推進協議会年次大会 予稿用テンプレート}
% 和文著書.
% 複数所属の場合, 右肩に片括弧数字を付与してください. Authblk には対応していません)
\author{森 鴎外$^{1)}$, 夏目 漱石$^{1),2)}$, 芥川 龍之介$^{2)}$}
% 和文所属. 所属ごとに改行 '\\' を入れてください
\affiliation{
1) 日本文学大学 情報センター \\
2) 日本文学大学 大学院文学研究科}
% コンタクトe-mail
\contactemail{mori@nitibun-u.ac.jp}
% 英文タイトル
\etitle{Manuscript Template for AXIES Annual Meeting}
% 英文著者. 和文所属と同様です
\eauthor{Ogai Mori$^{1)}$, Soseki Natsume$^{1),2)}$, Ryunosuke Akutagawa$^{2)}$}
% 英文所属. 和文所属と同様です
\eaffiliation{
1) Information Center, Nihon Bungaku Univ. \\
2) Graduate School of Literature, Nihon Bungaku Univ.}

% 概要の内容です
\begin{abstract}
大学ICT推進協議会年次大会の予稿用テンプレートです。大学ICT推進協議会年次大会の予稿用テンプレートです。大学ICT推進協議会年次大会の予稿用テンプレートです。大学ICT推進協議会年次大会の予稿用テンプレートです。	
\end{abstract}

\begin{document}
% \begin{document} 直後に \maketitle コマンドを実施してください.
\maketitle

\section{はじめに}

この サンプルファイル中の文の内容は、年次大会の論文作成と提出に関する情報を基に作成していますが、最新の内容は年次大会のウェブページで確認するようにしてください。 

\section{著者向け情報}
\subsection{論文の作成}

講演論文集はダウンロード版として出版します。1つの講演につき、A4判で2ページ以上8ページ以内の原稿を用意してください。詳細は論文作成要領を参照してください。

\subsection{論文の提出}

原稿は、PDFファイルをウェブ上でアップロードすることにより提出してください。論文ファイルの提出ページにアクセスして、指示に従い提出してください。提出期日は年次大会のウェブページで確認してください。

\section{論文作成要領}

講演論文集はダウンロード版のみです(冊子版やCD-ROM版はありません)。電子的な出版形態をとりますので、原稿の提出は基本的にPDF形式のファイルをアップロードすることにより提出してください。PDF形式で提出していただくことにより、文字はもちろんのこと図等も鮮明に印刷することが可能になります。 

読みやすく鮮明かつ正確な出版物とするために、以下の指示に従い間違いのないように執筆してください。なお、締切後の訂正は受けつけません。締切前であれば、訂正して再提出することができます。

\subsection{論文の書式について}
\begin{itemize}
	\item 原稿はA4判で2ページ以上8ページ以内です。ページ数は、読みやすさを損なわない範囲で、できるだけ少なくしてください。
	\item 原稿は日本語または英語で記述してください。
	\item ページの上端20mm、下端24mm、左右23mm以上の余白を設け、その中を原稿記入範囲としてください。
	\item 先頭ページの原稿記入範囲の上部に、論文タイトル、著者名、所属、メールアドレスの順に各項目を記入してください。また、タイトル、著者、所属情報は英文も併記してください。
	\item 本文の文字の大きさは10~11ポイントとしてください。
	\item ページは2段組を推奨します。
	\item 原稿にはページ番号を記載しないでください。
	\item 図表は本文中に埋め込んでください。
	\item 参考文献は以下の通りに記述してください。
	\begin{itemize}
		\item	雑誌の場合:著者名、タイトル、雑誌名、巻、号、ページ、発行年
		\item	書籍の場合:著者名、書名、参照ページ、発行所、発行年
	\end{itemize}
	\item	本文中で参考文献に関連する箇所には参考文献番号を付与してください。
\end{itemize}

\subsection{PDFファイルの作成について}

論文ファイルはPDF形式のファイルをウェブ上でアップロードすることにより、提出してください。提出できるファイル形式はPDFだけです。
\begin{itemize}
	\item ファイル名には拡張子(.pdf)を必ず付けてください。
	\item PDFファイルの大きさは、論文1件あたり10MBが上限です。
	\item 画像の解像度を必要以上に落とさないようにしてください。
	\item フォント情報を埋め込むと、意図した通りの仕上がりになるはずです。
	\item 編集不可・印刷不可などのセキュリティ設定を施さないでください。
\end{itemize}

\subsection{論文の提出}

年次大会ウェブページの指示に従ってファイルをアップロードすることにより、論文を提出してください。

\section{参考文献の書き方}

参考文献を記載するときの書式は、雑誌の場合\cite{refjournal}と書籍の場合\cite{refbook}それぞれ以下に従ってください。

\begin{thebibliography}{99}
	\bibitem{refjournal} 著者名、タイトル、雑誌名 巻、号、ページ、発行年.
	\bibitem{refbook} 著者名、書名、参照ページ、発行所、発行年.
\end{thebibliography}
\end{document}

% for vim cofiguration
% vim: set fileformat=unix fileencoding=utf8
